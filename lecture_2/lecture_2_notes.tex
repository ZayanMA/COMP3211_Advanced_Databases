\documentclass[11pt,a4paper]{article}

% --------- Layout / Typography ----------
\usepackage[margin=1in]{geometry}
\usepackage{microtype}
\usepackage{parskip}
\usepackage{hyperref}
\usepackage{amsmath,amssymb}

% --------- Lists / Tables ----------
\usepackage{enumitem}
\usepackage{booktabs}
\usepackage{tabularx}
\usepackage{array}

% --------- Nice boxes ----------
\usepackage[most]{tcolorbox}

% --------- Section spacing ----------
\usepackage{titlesec}
\titlespacing*{\section}{0pt}{1.0em}{0.5em}
\titlespacing*{\subsection}{0pt}{0.8em}{0.35em}
\titlespacing*{\subsubsection}{0pt}{0.6em}{0.25em}

% --------- List styling (INDENTED, readable) ----------
\setlist[itemize]{leftmargin=2.2em, label=\textbullet, itemsep=2pt, topsep=3pt}
\setlist[enumerate]{leftmargin=2.4em, itemsep=2pt, topsep=3pt}

% --------- Box styles ----------
\tcbset{
  colback=white,
  colframe=black!60,
  arc=2mm,
  boxrule=0.4pt,
  left=1.2mm,
  right=1.2mm,
  top=1mm,
  bottom=1mm
}
\newtcolorbox{defbox}{title=\textbf{Definition}, fonttitle=\bfseries}
\newtcolorbox{notebox}{title=\textbf{Note}, fonttitle=\bfseries}
\newtcolorbox{exbox}{title=\textbf{Example / Intuition}, fonttitle=\bfseries}
\newtcolorbox{opbox}{title=\textbf{Operations: what's meaningful?}, fonttitle=\bfseries}

% --------- Document ----------
\title{\textbf{COMP3211 Advanced Databases}\\[2pt]
\large Data Types and Operations --- Clean Notes}
\author{}
\date{}

\begin{document}
\maketitle

\section{Big Picture}
Advanced DBMSs must support \textbf{non-traditional data types} (temporal, spatial, multimedia) and provide suitable \textbf{operations} for querying them.
A key theme: \textbf{not every operation makes sense for every type}, so systems must define meaningful operations (and avoid misleading ones).

\section{Data Types in Databases}
Common types covered:
\begin{itemize}
  \item \textbf{Numeric} (integers, reals)
  \item \textbf{Character} (strings)
  \item \textbf{Temporal} (time-oriented data)
  \item \textbf{Spatial} (geometric data in 2D/3D)
  \item \textbf{Text} (documents, semi/unstructured text)
  \item \textbf{Image} (still images)
  \item \textbf{Audio \& Video} (multimedia streams/files)
\end{itemize}

\section{Operations on Data}
Typical operation families:
\begin{itemize}
  \item \textbf{Comparison:} equality, ordering, similarity
  \item \textbf{Arithmetic:} addition, multiplication, etc.
  \item \textbf{Fuzzy search / similarity:} approximate matching (especially for text/media)
  \item \textbf{Information retrieval queries:} e.g.\ documents containing a word; images containing a feature
\end{itemize}

\begin{defbox}
\textbf{Meaningful operation:} An operation is meaningful when its result has a clear, consistent interpretation for that data type.
If an operation is undefined or ambiguous, it should not be treated as a normal operator.
\end{defbox}

\begin{opbox}
\textbf{Units matter (weights vs numbers):}
\begin{itemize}
  \item $2\,\text{kg} + 2\,\text{kg}$ \ \ $\rightarrow$ meaningful (same unit, additive quantity)
  \item $2\,\text{kg} \times 2\,\text{kg}$ \ \ $\rightarrow$ generally \textbf{not meaningful} in normal DB semantics (unit becomes kg$^2$)
  \item $13 + 2\,\text{kg}$ \ \ $\rightarrow$ meaningless (incompatible kinds)
  \item $13 \times 2\,\text{kg}$ \ \ $\rightarrow$ meaningful (scale a quantity by a factor)
\end{itemize}

\textbf{Media types:}
\begin{itemize}
  \item ``Compare two images for equality'' is usually not meaningful (tiny changes break equality).
  \item ``Add two images'' is ambiguous unless the system defines a specific image-processing meaning.
\end{itemize}
\end{opbox}

\subsection{Ordering: Total vs Partial}
A common question for any type is: \textbf{is it ordered}, and what kind of order?

\begin{defbox}
\textbf{Total order:} any two values are comparable (for any $a,b$, either $a\le b$ or $b\le a$).\\
\textbf{Partial order:} some pairs may be incomparable (neither $a\le b$ nor $b\le a$ holds).
\end{defbox}

\begin{notebox}
Even if you \emph{can} impose an order (e.g.\ by ID), the key question is whether the order has \textbf{semantic meaning} or is just a convenience.
\end{notebox}

\section{Temporal Data}
Temporal data adds the time dimension to support questions such as:
\begin{itemize}
  \item Average price of product $X$ during 1995
  \item Month with the most copies sold of video $Y$
  \item Treatment history of patient $Z$
\end{itemize}

\subsection{Characteristics of Time}
Time can differ by:
\begin{itemize}
  \item \textbf{Structure:} linear; branching time (possible futures); directed acyclic graph; periodic/cyclic
  \item \textbf{Boundedness:} unbounded; bounded with an origin; bounded at both ends
\end{itemize}

\subsection{Time Density (Discrete / Dense / Continuous)}
Slides distinguish time models by how many time points exist between two points.

\begin{tabularx}{\linewidth}{@{}>{\bfseries}l X X X@{}}
\toprule
Model & Timeline resembles & Ordering property & Points between two points \\ \midrule
Discrete & Integers ($\mathbb{Z}$) & Total order & Finite number of chronons \\
Dense & Rational numbers ($\mathbb{Q}$) & Partial order & Infinite number of chronons \\
Continuous & Real numbers ($\mathbb{R}$) & Total order & Infinite number of chronons \\
\bottomrule
\end{tabularx}

\begin{defbox}
\textbf{Chronon:} the smallest representable time unit (a fixed period) used by a system (e.g.\ 1 second, 1 minute).
\end{defbox}

\subsection{Granularity}
Granularity = the resolution used when representing time.
\begin{exbox}
Event A at 11:00 and Event B at 15:00 on the same day:
\begin{itemize}
  \item If granularity = \textbf{1 day}, A and B occur in the same time unit $\Rightarrow$ no precedence is visible.
  \item If granularity = \textbf{1 minute}, A precedes B clearly.
\end{itemize}
\end{exbox}

\begin{notebox}
The slides also highlight a distinction between:
\begin{itemize}
  \item \textbf{Sequence:} order in which events are recorded/considered
  \item \textbf{Time:} actual temporal placement and distance
\end{itemize}
These can differ (e.g.\ logged later vs happened earlier).
\end{notebox}

\subsection{Storing Time in a Database}
A database fact/event can have multiple time notions:

\begin{itemize}
  \item \textbf{Valid time:} when the fact is true in the real world
  \item \textbf{Transaction time:} when the fact is current/stored in the DB and retrievable
  \item \textbf{Bitemporal:} storing \emph{both} valid and transaction time
\end{itemize}

\begin{exbox}
A correction scenario (intuition):
\begin{itemize}
  \item A fact could be valid from January, but only inserted into the DB in March.
  \item Valid time captures reality; transaction time captures DB history.
\end{itemize}
\end{exbox}

\section{Temporal SQL Extensions (TSQL)}
Extensions mentioned:
\begin{itemize}
  \item \textbf{WHEN clause} (temporal conditions)
  \item Timestamp retrieval
  \item Retrieval of temporally ordered information
  \item \texttt{TIME-SLICE} clause to specify a time domain
  \item Modified aggregate functions via \texttt{GROUP BY}
\end{itemize}

\subsection{WHEN Clause}
Format:
\[
\texttt{SELECT \{select-list\} FROM \{relations\} WHERE \{conditions\} WHEN \{temporal clause\}}
\]

Temporal comparison operators include:
\begin{itemize}
  \item BEFORE / AFTER
  \item PRECEDES / FOLLOWS
  \item DURING
  \item EQUIVALENT
  \item ADJACENT
  \item OVERLAPS
\end{itemize}

\begin{notebox}
These operators relate to \textbf{interval reasoning} (as in Allen's Interval Calculus): rather than comparing single timestamps, you compare \emph{interval relationships} (overlap, adjacency, containment, etc.).
\end{notebox}

\section{Spatial Data}
Spatial data represents objects in space.

\subsection{Spatial Data Types}
Types listed:
\begin{itemize}
  \item Points
  \item Regions
  \item Boxes
  \item Quadrangles
  \item Polynomial surfaces
  \item Vectors
\end{itemize}

\subsection{Common Spatial Operations}
Operations listed:
\begin{itemize}
  \item Length / distance (where defined)
  \item Intersection
  \item Containment
  \item Overlap
  \item Centre computation
\end{itemize}

\subsection{Applications \& Properties of Interest}
Main application areas:
\begin{itemize}
  \item Computer Aided Design (CAD)
  \item Computer generated graphics
  \item Geographic Information Systems (GIS)
\end{itemize}

Properties of interest:
\begin{itemize}
  \item \textbf{Connectivity} (what is linked/connected?)
  \item \textbf{Adjacency} (what touches what?)
  \item \textbf{Order} (arrangement/sequence in space)
  \item \textbf{Metric relations} (distances, angles, areas)
\end{itemize}

\subsection{Why Spatial DB Performance is Hard (from slides)}
\begin{itemize}
  \item Objects can be highly complex
  \item Data volumes can be very large
  \item Real-time constraints may apply
  \item Performance is not easy to achieve
  \item Often accessed via specialised graphical front-ends (operator skill matters)
  \item Query processing may not use standard SQL
\end{itemize}

\section{Multimedia Data}

\subsection{Text Data}
Text may be:
\begin{itemize}
  \item Already machine-readable (word processors, spreadsheets, etc.)
  \item Extracted via OCR
\end{itemize}

Key issue: text is \textbf{essentially unstructured} $\Rightarrow$ retrieval needs an index:
\begin{itemize}
  \item Human-built index, or
  \item Automatically built \textbf{inverted list} (index of significant words $\rightarrow$ documents containing them)
\end{itemize}

\begin{defbox}
\textbf{Inverted index / inverted list:} maps each word (term) to the set of documents that contain it, enabling fast queries like “find all documents containing word $w$”.
\end{defbox}

Markup languages add structure:
\begin{itemize}
  \item HTML (web)
  \item XML / SGML (portable documents with structured data; can define new markup languages)
\end{itemize}

DB support mentioned:
\begin{itemize}
  \item \textbf{CLOBs} (Character Large Objects) for storing text documents
  \item Text search and retrieval facilities
\end{itemize}

\subsection{Document-Style Queries (Motivation)}
Typical useful queries:
\begin{itemize}
  \item Legal documents concerning client “Jones”
  \item Suspects with false teeth who have been interviewed
  \item Articles on “databases”
\end{itemize}

\subsection{Image Data}
Examples:
\begin{itemize}
  \item X-rays, maps, photographs
\end{itemize}

Storage:
\begin{itemize}
  \item Stored as \textbf{BLOBs} (Binary Large Objects)
  \item No attached semantics by default (the DB stores bits, not meaning)
\end{itemize}

Image databases need support for:
\begin{itemize}
  \item Image analysis and pattern recognition
  \item Image structuring and understanding
  \item Spatial reasoning and image information retrieval
\end{itemize}

\begin{defbox}
\textbf{QBIC (Query By Image Content):} retrieve images using content features (e.g.\ colour/texture/shape) rather than only filenames/labels.
\end{defbox}

\subsection{Audio Data}
Digitised audio:
\begin{itemize}
  \item Formats: WAV, MP3
  \item Consumes large storage; compression commonly used
\end{itemize}

MIDI:
\begin{itemize}
  \item More compact than digitised audio
  \item Stored as instruction sequences (e.g.\ \texttt{Note\_On}, \texttt{Note\_Off}, \texttt{Increase\_Volume})
  \item Interpreted by a synthesiser
\end{itemize}

\subsection{Video Data}
Video characteristics:
\begin{itemize}
  \item Extremely space-hungry
  \item Stored as a sequence of frames (each frame can be $>$ 1MB)
  \item Playback typically 24--30 fps
\end{itemize}

Audio-video integration:
\begin{itemize}
  \item Interleaved file structures coordinate time sequencing
  \item Examples: Microsoft AVI, Apple QuickTime
\end{itemize}

\section{Quick Consolidation (What to Remember)}
\begin{itemize}
  \item Different data types $\Rightarrow$ different meaningful operations.
  \item Ordering matters: total vs partial order; semantic vs convenience order.
  \item Temporal: structure, boundedness, density (discrete/dense/continuous), granularity, valid vs transaction time.
  \item Spatial: specialised types + geometry operations; large/complex data makes performance hard.
  \item Multimedia:
    \begin{itemize}
      \item Text needs indexing (inverted index); markup adds structure (HTML/XML).
      \item Images are BLOBs with no semantics unless analysed; QBIC = content-based retrieval.
      \item Audio/video are storage-heavy; compression and timing/sync are key.
    \end{itemize}
\end{itemize}

\end{document}
